\starttext 


\title \bf  \rm EXPÉRIMENTATIONS AUTOUR DE L'ÉCHELLE

\it Anaïs . Jordane . Alexandra


Le principe de grille définit un espace dans lequel se construit la lettre, la page, le chapitre, le chemin de fer.

En s'appuyant sur des grilles créées avec "nos petits robots qui dessinent" nous voulons essayer de mettre en avant les structures de ces éléménts constitutif du livre à travers ces différentes échelles. Fractalité du livre ? Qu'est-ce qui définit l'unité ? Y'a-t-il une unité pour chaque changement d'échelle ? 
Les formes des grilles créés par les "petits robots qui dessinent" ne dépendent pas de nous mais de nombreux facteurs liés au contexte dans lequel dessinent les robots. En partant de leur accidents de parcours, des formes non maitrisées qui naissent, on obtient des grilles que l'on n'aurait pas construite de la même manière sans ces robots. Nous posons donc ces grilles comme une contrainte qui servira de base à notre recherche, à notre dessin de caractère, de mise en page...

\externalfigure [http://ospublish.constantvzw.org/images/var/resizes/Up-Pen-Down-December/P1130050.JPG] [width=10cm] \externalfigure [http://ospublish.constantvzw.org/images/var/resizes/Up-Pen-Down-December/conv_IMG_0949.jpg][width=10cm] 

\externalfigure[https://raw.github.com/no-feature/GRoland329/master/01.jpg][width=10cm] \externalfigure[https://raw.github.com/no-feature/GRoland329/master/02.jpg][width=10cm]

\externalfigure[https://raw.github.com/no-feature/GRoland329/master/03.jpg][width=10cm]



.......Bon mais tout ça c'était avant !



\bf \CAP {Introduction PROJET : GRoland 329}

\CAP {BRAINSTORMING}

\setupcolumns[n=3, rule=on, ntop=4]
   
       \startcolumns
                                    \startitemize 
                                         \item {Nos petits robots qui dessinent ont produit des sortes de cartographies}
                                         \item {Transition d'échelles}
                                         \item {Observation d'irégularités et de formes intéressantes dans le trait}
                                         \item {Reproduction}
                                         \item {Agrandissement}
                                         \item {Déformation}
                                         \item {Notion temps dans la forme du trait (pointillés, épaisseur...)}
                                         \item {Fabrication d'un pantographe avec un moteur (on a remarqué que le moteur permettait un tracé plus fluide et qui ressemblait au 
                                                    plotter notament dans l'observation des points des lignes)}
                                         \item {Pantographe : prolongement du geste, déformation du geste par les frottements dans les articulations du Groland, transition 
                                                   d'échelle, grossissement du dessin original.}
                                   \stopitemize
          \stopcolumns

Utilisation du pantographe comme un outils de reproduction, et peut être envisager en outils de double productions (polygraphe).

une écriture en deux temps (polygraphe) créer un caractère d'un côté et voir ses défauts 

Investir Groland  d'une mission :  celle de reproduire un aplhabet pour mettre en avant la dégradation, modification, transformation du trait, mettre en exergue les variations de formes dû au changement d'échelle.

Face à l'utilisation quotidienne d'outils de reproduction de plus en plus performant, nous souhaitons expérimenter l'outils du pantographe où l'accident survient facilement grâce au poids et aux frottements des matériaux utilisés (boix, feuille, moteur, roulettes, écrous, vis). 

Nous avons d'abord proposer un outil de reproduction et d'agrandissement. Ainsi la machine, par de multiple accidents, produit des copies déformant le dessin original.

En s'appuyant sur l'expérience du pantographe, nous cherchons à questionner la transition d'échelle et la notion de reproductibilité. A l'heure d'une reproductibilité de masse accrue, qu'en est-il de la présence du geste dans ces transitions d'échelle ? Comment le pantographe réintroduit-il l'accident à la manière du plotter ? Et comment l'accident laisse place à une trace toujours plus riche et exploitable ? 

La fractalité de la trace, à travers cette démarche, questionne la notion d'échelle (infiniment grand et infiniment petit). Comment la ligne d'une lettre peut-elle par un agrandissement devenir une possible ligne de cartographie ?

Outre cette notion d'agrandissement de la ligne, la translation proportionnelle que l'on obtient grâce à notre pantographe, nous amène à transposer le dessin d'une main à l'échelle du corps. Le dessin prend des proportions qui lui donne un nouveau statut car elle "rentre dans l'espace". 

\externalfigure [http://ospublish.constantvzw.org/images/var/albums/Up-Pen-Down-December/IMG_1840.JPG][width=10cm] 

\externalfigure [http://ospublish.constantvzw.org/images/var/resizes/Up-Pen-Down-December/IMG_1842.JPG][width=10cm] 

\externalfigure [http://ospublish.constantvzw.org/images/var/resizes/Up-Pen-Down-December/IMG_1843.JPG][width=10cm] 

\externalfigure [http://ospublish.constantvzw.org/images/var/resizes/Up-Pen-Down-December/IMG_2063.JPG][width=10cm] 

\externalfigure [http://ospublish.constantvzw.org/images/var/resizes/Up-Pen-Down-December/IMG_2064.JPG][width=10cm] 




\stoptext

