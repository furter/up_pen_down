\startcomponent *
\environment styles
\environment typo
\enableregime[utf-8]
	\starttext 
	    \startalignment[right]
	\chapter{Dialogue}
\page
\setupinterlinespace[12pt]   
\section {Prémices du projet}

À partir d'une réflexion sur les outils et les médias apportés par OSP, nous avons rencontré plusieurs lectures. \logosssssb {Le geste d'écrire}, \logossb {de Vilém Flusser, a été pour nous le moteur de notre questionnement sur la relation entre l'homme et la machine. Après un exercice de mise en commun des références, deux artistes ont, pour nous, fait sens. Puisant dans le travail sur les frontières de Francis Alÿs} \logosssssb { Sometimes doing something poetic can become political and sometimes doing something political can become poetic} \logossb {et le projet }\logosssssb {mesuRages} \logossb {d'Orlan, nous avons voulu mettre en avant l'utilisation du corps comme outil. De là découle pour nous plusieurs questions élémentaires, comme « Qu'est-ce qu'une machine ? Quelle est la différence entre nous et la machine ? Est-ce que la sensibilité a une place légitime dans un rapport homme-machine ? »}
 
 \externalfigure[../IMAGES/DG/Alys.png][width=8cm]
 \page 
 \externalfigure[../IMAGES/DG/Orlan.png][width=8cm]
 
\subsubject{Le geste d'écrire}
\startquotation
\citations {La structure du geste est linéaire. Mais il s'agit d'une linéarité spécifique. On commence, en thèse; par le coin supérieur à gauche de la surface, on fait une ligne jusqu'à ce qu'on arrive au coin supérieur à droite, on saute à gauche pour recommencer le geste un peu plus en bas, et on répète ce mouvement jusqu'au coin inférieur droit. Il s'agit d'une structure apparemment accidentelle : elle a été imposée sur le geste par les accidents de notre histoire. Elle pourrait être différente, et, en effet, elle l'est dans d'autres civilisations. Néanmoins : cette structure-là, qui est le résultat d'accidents aussi méprisables comme c'est la qualité de la boue en Mésopotamie, ordonne toute une dimension de notre être-dans-monde: elle ordonne nos pensées linéaires, logiques, historiques, scientifiques. Car nous sommes programmés pour ce type de pensées par notre écriture, et, inversement, ces pensées sont programmées pour être écrites selon la structure que je viens de décrire. Le moindre changement dans cette structure changerait, sans doute, ce type de pensée. Mais, bien sûr, l'inverse est aussi vrai : tout changement structurel dans nos pensées implique un changement dans la structure de l'écriture. Peut-être est-ce en train d'arriver à présent.}
\stopquotation
\logosssssb {Le geste d'écrire,} \logosssb {Vilém Flusser.}



\subsubject {Le Xerox et l'infinity}

\startquotation
\citations {Ce qui distinguera toujours le fonctionnement de l'homme et celui des machines, même les plus intelligentes, c'est l'ivresse de fonctionner, le plaisir. Inventer des machines qui aient du plaisir, voilà qui est heureusement encore au delà des pouvoirs de l'homme. Toutes sortes de prothèses peuvent aider à son plaisir, mais il ne peut en inventer qui jouiraient à sa place. Alors qu'il en invente qui travaillent, « pensent » ou se déplacent mieux que lui ou à sa place, il n'y a pas de prothèse, technique ou médiatique, du plaisir de l'homme, du plaisir d'être homme. Il faudrait pour celà que les machines aient une idée de l'homme, qu'elles puissent inventer l'homme, mais pour elles il est déjà trop tard, c'est lui qui les a inventées. C'est pourquoi l'homme peut excéder ce qu'il est, alors que les machines n'excéderont jamais ce qu'elles sont. Les plus intelligentes ne sont exactement que ce qu'elles sont, sauf peut-être dans l'accident et la défaillance, qu'on peut toujours leur imputer comme un désir obscur. 
Elles n'ont pas ce surcroit ironique de fonctionnement, cet excès de fonctionnement en quoi consistent le plaisir ou la souffrance, par où les hommes s'éloignent de leur définition et se rapprochent de leur fin. Hélas pour elle, jamais une machine n'excède sa propre opération, ce qui peut-être explique la mélancolie profonde des {\em computers}… toutes les machines sont célibataires. (pourtant la récente irruption des virus électroniques offre une anomalie remarquable : on dirait qu'il y a un malin plaisir des machines à amplifier, voire à produire des effets pervers, à excéder leur finalité par leur propre opération. Il y a là une péripétie ironique et passionnante. Il se peut que l'intelligence artificielle se parodie elle même dans cette pathologie virale, inaugurant par là une sorte d'intelligence véritable.)}
\stopquotation
\logosssssb {Le Xerox et l'infinity,} \logosssb {Jean Baudrillard.}
\page
\subsubject{Une histoire d'amour ratée \crlf avec une machine à écrire}

\startquotation
\citations{Nos outils d’écriture travaillent aussi nos pensées", a écrit Nietzsche. "La technologie est retranchée dans notre histoire", a dit Heidegger. Mais l’un a écrit cette phrase à propos de la machine à écrire sur une machine à écrire quand l’autre décrivait (dans un allemand ancien magnifique) l’essence de la machine à écrire. Voilà pourquoi c’est Nietzsche qui fut à l’origine de la transvaluation des valeurs avec sa phrase philosophiquement scandaleuse sur la technologie des médias, ‘les hommes ne sont peut-être que des machines pensantes, écrivantes et parlantes. En 1882, aux êtres humains, à leurs pensées et à la figure de l’auteur se sont substitués terme à terme les deux sexes, le texte et l’appareillage de l’écriture à l’aveugle. Premier philosophe mécanisé, Nietzsche fut aussi le dernier. Le texte dactylographié avait pour nom, si l’on en croit la peinture de Klapheck, Volonté de puissance.}
\stopquotation
\logosssssb {Une histoire d'amour ratée avec une machine à écrire, \crlf }\logosssb {Friedrich Kittler.}


\page
\section {Premieres expérimentations :\crlf intervention sur les machines}	
\externalfigure [../IMAGES/DG/croquis1.jpg][width=5cm]


\externalfigure [../IMAGES/DG/dialogue5.jpg][width=5cm]
\blank
\subsubject {Utilisation de nouveaux outils}

\externalfigure [../IMAGES/DG/croquis2.jpg][width=5cm]


\externalfigure [../IMAGES/DG/tof-experimentatation2.jpg][width=7.5cm]		
		\page

		
		\setupcolumns [n=2]
		\startcolumns 

\page


\externalfigure [../IMAGES/DG/2.jpg][width=5cm]
\logossssb{« Souvenir d'un outilleur », \crlf d'après Pierre Bézier } \blank

\externalfigure [../IMAGES/DG/3.jpg][width=5cm]

\logossssb{« Dialogues »} \blank

\externalfigure [../IMAGES/DG/4.jpg][width=5cm]

\logossssb{« Dialogues »} \blank

\externalfigure [../IMAGES/DG/5.jpg][width=5cm]

\logossssb{« Écrire c'est réaliser ses pensées »} \blank

\externalfigure [../IMAGES/DG/6.jpg][width=5cm]

\logossssb{« ??? »} \blank

\externalfigure [../IMAGES/DG/7.jpg][width=5cm]

\logossssb{« Les hommes deviendrons des machines pensantes et parlantes »} \blank

\externalfigure [../IMAGES/DG/8.jpg][width=5cm]

\logossssb{« Quand sa machine s'effondra », \crlf d'après Friedrich Kittler, Une histoire d'amour ratée avec une machine à écrire} \blank

\externalfigure [../IMAGES/DG/9.jpg][width=5cm]

\logossssb{« ??? »} \blank

\externalfigure [../IMAGES/DG/10,5.jpg][width=5cm]

\logossssb{« Il est une chose »} \blank

\externalfigure [../IMAGES/DG/5.jpg][width=5cm]

\logossssb{« Écrire c'est réaliser ses pensées »} \blank

\externalfigure [../IMAGES/DG/1.jpg][width=5cm]

\logossssb{« Accident de notre histoire »} \blank

\externalfigure [../IMAGES/DG/essai-4.jpg][width=5cm]

\stopcolumns		

\page

\section {Sur le langage}

\subsubject {Début d'un langage}
		
\logossb {Notre langage se développe en fonction de l'outil que nous utilisons et de l'objectif à atteindre, et si « nos outils d'écritures travaillent aussi nos pensées \footnote{Nietzsche} », nous pouvons aussi dire que nos outils déterminent notre langage. Au cours de nos expérimentations, notre condition humaine provoque un langage en écho avec celle-ci : des phrases spontanées, des intonations, des mouvements, des adaptations, à contrario d'un langage plus cartésien, dénué de sensibilité et ne laissant pas forcément place à l'erreur. Ce même langage évolue au fil de l'expérience, il se transforme, s'enrichit et n'est pas figé. Notre langage comporte des accidents et le résultat nos expériences. Dépendant de l'attitude de l'autre, nous concluons que notre langage est le fruit d'une corrélation entre deux personnes. Il s'agit d'un véritable dialogue, surgissant naturellement de nos tentatives, qui s'anime et se précise au cours de la pratique.} 



\subsubject{Observations}

\subsubject {Introduction d'un langage \crlf intuitif/inconscient/sensible \crlf dans notre processus d'écriture}
\setupinterlinespace[14pt]   
\logossb {Les actions \blank
Celui qui tient le stylo = + passif \crlf
- action de pression/souplesse du stylo = \crlf plus fort /moins fort/moins raide…\crlf
- serrage du stylo = + ou -\crlf
- concentration = diminution et augmentation \crlf
- relâchement du corps = pauses obligatoires \crlf
- fermer les yeux \crlf
Celui qui tient la table = + dans l'action \crlf
- vitesse dans le maniement de l'outil = \crlf plus vite/moins vite \crlf
- concentration = diminution et augmentation/\crlf concentration extrême \crlf
- force = augmentation / diminution \crlf
- relâchement du corps = pauses obligatoires \crlf
- éviter le corps de l'autre = redéfinir l'espace \crlf autour de soi \crlf
- lier les lettres par le bas ou par le haut \crlf}

\subsubject{Le langage de la table qui roule}

\logossb {« Ferme les yeux ! » \crlf
« Concentre-toi ! »\crlf
« Serre-moins le stylo ! »\crlf
« Attention à ton ventre ! » « Je vais tourner »\crlf
« J'ai mal au bras »\crlf
« On fait une pause ? »\crlf
« J'en peux plus ! »\crlf
« Pfiouuuu »\crlf
« C'est lourd à la fin… »\crlf}



\subsubject{Langage de la table qui tourne}

\logossb {« Ferme les yeux ! »\crlf
« Fais des mouvements plus longs ! »\crlf
« Fais des mouvements plus courts ! »\crlf}

\page
—
\page
    \stopalignment

	\stoptext
\stopcomponent
