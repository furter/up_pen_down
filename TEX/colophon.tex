\startcomponent *
\title {Colophon}
\startext
"Les oiseaux migrateurs doivent parcourir de très longues distances, dans des conditions parfois difficiles. Ainsi, il est important pour eux d'optimiser leur déplacement en termes d'énergie dépensée. Les oies sauvages adoptent des formations en V qui leur permettent d'étendre leur distance de vol de près de 70\%, car chaque oiseau prend l'aspiration de son prédécesseur, comme le font les cyclistes.
Le prix à payer est une perte en vitesse, puisqu'un individu seul vole en moyenne 24\% plus vite qu'une volée." http://fr.wikipedia.org/wiki/Intelligence_collective

"La sociocratie est un mode de prise de décision et de gouvernance qui permet à une organisation de se comporter comme un organisme vivant et de s'auto-organiser. L'objectif premier est de développer la co-responsabilisation des acteurs et de mettre le pouvoir de l'intelligence collective au service du succès de l'organisation. La méthode sociocratique est fondée sur le concept plus aucune objection argumentée d'aucune personne." http://fr.wikipedia.org/wiki/Sociocratie

Environ à la médiane du workshop qui a produit ce livret, une session autour d'une table ronde s'est déroulé sous un mode relativement sociocratique et a défini les grandes lignes de l'organisation à mettre en place pour terminer le document dans les délais. On ne saura sans doute jamais si cette expérience particulière a permit certains huilages de roue de conception ou si le temps nécessaire à l'expérimenter aurait pu servir avantageusement à d'autres tâches. Mais elle a en tout cas été l'occasion d'articuler les mots alibi, autorisation, contrainte, critique, signer, sincère, entre autres et dans l'ordre alphabétique.

Up Pen Down
Atelier des étudiants en 4e année Design Graphique à Valence, avec Open Source Publishing (Alexandra Rio, Alice Jauneau, Anaïs Alauzen, Camille Chatelaine, David Vallance, Éléonore Jasseny, Gwenaël Fradin, Jordane Cals, Lorène Ceccon, Margot Baran et Vincent Duché avec Alexandre Leray, Eric Schrijver, Femke Snelting, Gijs De Heij, John Haltiwanger, Ludi Loiseau, Pierre Huyghebaert, Pierre Marchand, Stéphanie Vilayphiou)
Le texte d'intention initial s'énonçait comme suit :
Notre intervention se place autour du “trait”, dans le sens de “chemin” (path en anglais) par opposition et en dialogue avec la notion de forme. Point de départ offrant plusieurs perspectives - celles du dessin, de la typographie, de la cartographie - ce “chemin” pose aussi la question des outils - en particulier numériques - et de leur relation avec le langage visuel.
En tant que constructions techniques, intellectuelles et culturelles, les logiciels incarnent des conceptions spécifiques des objets qu’ils manipulent (ici, le trait et la forme). Le format PostScript, par exemple, décrit les glyphes par leur contour plutôt que par leur squelette, amenant ainsi à une conception spécifique de la lettre, mais il existe d'autres formats de description moins connus qui prennent l’approche inverse.
Le logiciel libre, par la disponibilité du code mais surtout par l'ouverture des discussions relatives à l’élaboration des outils (par exemple par le biais des listes de discussion, ou de l'utilisation de logiciels de gestion des versions), permet de mieux saisir les implications des outils dans le travail du designer.
Une large série d'outils a été utilisés, dont les plus mémorables sont :

ConTeXt

FontForge

Git

Inkscape

Gimp

Vim

Github

TexShop

Virtualbox

Chiplotle

\stoptext
\stopcomponent
