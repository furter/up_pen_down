\startcomponent dr
\enableregime[utf-8] 
\starttext
\part {Dr._Karl_Hess_Keffer}


\definepapersize [format] [width=110mm, height=170mm]
 \setuppapersize [format] [format]
 
 \setupexternalfigures[maxwidth=9cm]
 
 \setupbodyfont[10pt]
      \definebodyfontenvironment[default][a=1.5, b=2, c=4, d=6, xx=0.5 ]
      
      
\subject Étude de la pensée collective par le Dr Karl Hess Keffer      

Restitution d'une archive des expériences d'outils réalisées par le Dr. Hess Keffer, neuro-scientifique.
Ces outils tentent de trouver un lieu d'expression de la pensée collective. 

Dr. Karl Hess Keffer a fait de nombreuses recherches sur l'activité cérébrale. 
Ces expériences mettent en jeu différents aspects de la collaboration. 

Cette recherche utopique n'a jamais trouvé de réponse ultime, car chaque outil met en lumière une faille: l'impossibilité d'un systématique consensus commun. 

Après 35 ans de recherches, le Dr Karl Hess Keffer abandonna son étude vaine. Actuellement, il travaille sur l'instinct grégaire des animaux. 

\note [footnote] photo 1

\externalfigure[../IMAGES/DR/Experience1.jpg][width=17cm]
\footnotetext[footnote]{legende}

\placefootnotes

\externalfigure[../IMAGES/DR/experience4.jpg][width=17cm]
–Expérience 2

\externalfigure[../IMAGES/DR/experience5.jpg][width=17cm]
–Expérience 3

\externalfigure[../IMAGES/DR/xp3.jpg][width=17cm]
–Expérience 4

\externalfigure[../IMAGES/DR/xp6.jpg][width=17cm]
–Expérience 5

\externalfigure[../IMAGES/DR/xp2.jpg][width=17cm]
–Expérience 6

\externalfigure[../IMAGES/DR/outils_experience1.jpg][width=15cm]
\blank

\externalfigure[../IMAGES/DR/outils_experience6.jpg][width=15cm]
–Matériaux

\externalfigure[../IMAGES/DR/scan1.jpg][orientation=90, width=12cm]

\externalfigure[../IMAGES/DR/scan2.jpg][orientation=90, width=12cm]

\externalfigure[../IMAGES/DR/scan3.jpg][orientation=90, width=12cm]

\externalfigure[../IMAGES/DR/scan4.jpg][orientation=-90, width=12cm]

\externalfigure[../IMAGES/DR/scan5.jpg][orientation=-90, width=12cm]

\externalfigure[../IMAGES/DR/scan6.jpg][orientation=90, width=12cm]

 Mes recherches sur la pensée collective menées ces 35 dernières années n'auront pas aboutis à une ou des réponses mais à des reconsidérations.

Mon intention était de mettre en place des outils proposant un espace d'expression égal à chaque utilisateur. Je constate aujourd'hui que mes tentatives invitent plus à des rapports de force qu'au dialogue, laissant comme trace des formes indécises, illisibles, vestiges de problèmes non résolus. 

Je réalise maintenant qu'il est question de design. Les formes induisent des comportements, des utilisations. Mais quelle forme pour quelle pensée ? Ces outils comportent finalement une part d'ironie dans leur volonté utopique d'obtenir une seule et unique forme réflexive de plusieurs utilisateurs, coupant, paradoxalement, le dialogue au profit du corps et de la domination.


Dr. Karl Hess Keffer

\stoptext

\stopcomponent

