\startcomponent *
\environment typo
\enableregime[utf-8] 
\starttext

\title {Étude de la pensée collective par le
 Dr Karl Hess Keffer}
\page
 \setupexternalfigures[maxwidth=7cm]
          \startalignment[right]

      
\subsubject {Étude de la pensée collective}  

\blank 
\blank
\blank

Restitution d'une archive des expériences d'outils réalisées par le Dr. Hess Keffer, neuro-scientifique.
Ces outils tentent de trouver un lieu d'expression de la pensée collective. 

Dr. Karl Hess Keffer a fait de nombreuses recherches sur l'activité cérébrale. 
Ces expériences mettent en jeu différents aspects de la collaboration. 

Cette recherche utopique n'a jamais trouvé de réponse ultime, car chaque outil met en lumière une faille: l'impossibilité d'un systématique consensus commun. 

Après 35 ans de recherches, le Dr Karl Hess Keffer abandonna son étude vaine. Actuellement, il travaille sur l'instinct grégaire des animaux. 


\page 

\externalfigure[../IMAGES/DR/Experience1.jpg][width=8cm]
{\tfxx –Le hamac
 \blank L'expérience du hamac réside dans le port du troisième individu (le plus léger ou le plus petit). Autour de sa taille est attaché le pen-ceinture, avec lequel les deux autres qui tiennent le troisième individu de bout en bout doivent écrire. Physiquement éprouvant et ridicule, le résultat n'est pas probant.}

\externalfigure[../IMAGES/DR/scan3.jpg][width=8cm]

\externalfigure[../IMAGES/DR/experience4.jpg][width=8cm]
{\tfxx –Spiritisme
 \blank Si certains illuminés constatent la mouvance d'un verre alors que chacun des protagonistes à son doigt dessus, ici, le stylo n'est pas assez stable, ou les individus n'ont pas assez de dextérité.}

\externalfigure[../IMAGES/DR/scan4.jpg][width=8cm]

\externalfigure[../IMAGES/DR/experience5.jpg][width=8cm]
{\tfxx –L'attache
 \blank Nous constatons ici le mal être des cobayes lors de l'attachement de leur poignet. Le contact physique rend difficile la concentration de tous, et les plus douillets se plaignent d'une mauvaise circulation. La pensée commune est impraticable dans ce contexte.}

\externalfigure[../IMAGES/DR/scan5.jpg][width=8cm]

\externalfigure[../IMAGES/DR/xp3.jpg][width=8cm]

{\tfxx –Kidnapping \blank
Chacun est lié par les coudes. Utilisation du pen-ceinture. Les individus n'aperçoivent pas leur outil. Un tendons d'Achille est finalement abîmé.}


\externalfigure[../IMAGES/DR/scan1.jpg][width=8cm]

\externalfigure[../IMAGES/DR/xp6.jpg][width=8cm]

{\tfxx –Trio en fil de laine \blank
Ce premier outil met en évidence sa fragilité. Le stylo n'est pas praticable pour les individus. (voir "Trio en bois")}


\externalfigure[../IMAGES/DR/scan6.jpg][width=8cm]

\externalfigure[../IMAGES/DR/xp2.jpg][width=8cm]

{\tfxx –Trio en bois \blank
Ce second outil, conçu après l'échec du "Trio en fil de laine" (voir "Trio en fil de laine") a une meilleure prise en main. Les protagonistes forcent chacun sur leur partie pour diriger le stylo. L'entente n'est pas très évidente entre eux. Cela rend la journée d'expériences difficile.}

\externalfigure[../IMAGES/DR/scan2.jpg][width=8cm]
        \stopalignment[right]


\page 

Chers lecteurs,
\blank
\blank

Mes recherches sur la pensée collective menées ces 35 dernières années n'auront pas aboutis à une ou des réponses mais à des reconsidérations.
 \blank

Mon intention était de mettre en place des outils proposant un espace d'expression égal à chaque utilisateur. Je constate aujourd'hui que mes tentatives invitent plus à des rapports de force qu'au dialogue, laissant comme trace des formes indécises, illisibles, vestiges de problèmes non résolus. 

\blank

Je réalise maintenant qu'il est question de design. Les formes induisent des comportements, des utilisations. Mais quelle forme pour quelle pensée ? Ces outils comportent finalement une part d'ironie dans leur volonté utopique d'obtenir une seule et unique forme réflexive de plusieurs utilisateurs, coupant, paradoxalement, le dialogue au profit du corps et de la domination.

\blank
\blank
\blank


Dr. Karl Hess Keffer
\blank
\externalfigure[../IMAGES/DR/Signature.png][width=3cm]

\stoptext

\stopcomponent

