\startproduct dialogue

	\starttext 
		

\title{Dialogue}

 À partir d'une réflexion sur les outils et les médias apportée par OSP, nous avons rencontré plusieurs lectures. Le geste d'écrire, de Vilém Flusser, a été pour nous le moteur de notre questionnement sur la relation entre l'homme et la machine. Après un exercice de mise en commun des références, deux artistes ont, pour nous fait sens. Puisant dans le travail sur les frontières de Francis Alÿs (Sometimes doing something poetic can become political and sometimes doing something political can become poetic) et le projet mesuRages d'Orlan, nous avons voulu mettre en avant l'utilisation du corps comme comme outils. De là découle pour nous plusieurs questions élémentaires, comme qu'est-ce qu'une machine ? Quelle est la différence entre nous et la machine ? Ainsi que Est-ce que la sensibilité à une place légitime dans un rapport homme-machine ?


Sur le langage


Notre langage se développe en fonction de l'outil que nous utilisons et de l'objectif à atteindre, et si « nos outils d'écritures travaillent aussi nos pensées(1) », nous pouvons aussi dire que nos outils déterminent notre langage. Au cours de nos expérimentations, notre condition humaine provoque un langage en écho avec celle-ci : des phrases spontanées, des intonations, des mouvements, des adaptations, à contrario d'un langage plus cartésien, dénué de sensibilité et ne laissant pas forcémment place à l'erreur. Ce même langage évolue au fil de l'expérience, il se transforme, s'enrichie et n'est pas figé. Notre langage comporte des accidents et le résultat nos expériences aussi. Dépendants de l'attitude de l'autre, nous concluons que notre langage est le fruit d'une corrélation entre deux personnes. Il s'agit d'un véritable dialogue, surgissant naturellement de nos tentatives, qui s'anime et se précise au cours de la pratique. 

(1) Nietzsche




	\stoptext
\stopproduct
