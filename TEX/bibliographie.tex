\startcomponent *
\enableregime[utf-8] 
\starttext

\title {Bibliographie}

\subsubject{Textes:}


\startitemize 
                                         \item {Fernand Baudin, {\it L'écriture au tableau noir}, Retz, 1984}
                                         \item {Walter Crane, {\it Line and Form}, 
                                         
                                         http://www.gutenberg.org/files/25290/25290-h/25290-h.html}
                                         \item {Tim Ingold, {\it Une brève histoire des lignes}, Zones Sensibles, 2011}
                                         \item {Vilém Flusser, {\it Le geste d'écrire}, Flusser Studies, n°8, Mai 2009, 
                                         
                                         www.flusserstudies.net/pag/08/le-geste-d-ecrire.pdf}
                                         \item {Donald Knuth, {\it Le concept de MetaFonte}, Visible Language, issue 16.1, January 1982 dans Communication et Language, n°55, 1983, pp 40-53}
                                         \item {Donald Knuth, {\it Lessons learned from MetaFont}, Visible Language, issue 19.1, décember 1985}
                                         \item {Gerrit Noordzij, {\it The Stroke - theory of writing}, Hyphen Press, 2006}
                                         \item {Femke Snelting, {\it Scenes of Pressures and relief}, 2009, 
                                         
                                         http://snelting.domainepublic.net.textes/pressure.txt}
                                         
                                         \item {Seymour Papert, {\it Turtle Geometry: A Mathematics Made for Learning}, p.31-47}
                                         \item {Femke Snelting, {\it Objects and Curves}, Scenes of Pressures and relief, 2009, non publié. traduction française par OSP} 
                                         \item {François Rappo \& Jürg Lehni, {\it Typeface as Program}} 
                                         \item {François Rappo, {\it How long should I work on that curve?}, in the context of 
                                         
                                         {\it Types We Can Make}} 
                                        
                                   \stopitemize
                                   
                                   
     \page                              
      \subsubject { Articles sur le web}                            
                                   \startitemize 
                                   
                                         \item {PmWiki (ERG-LIBRE), http://ludi.be/erg-libre/index.php?n=PmWiki.VirtualBox}
                                         \item {{\it Labanotation : système de notations/partitions pour danseurs/performeurs }
                                         
                                         http://sarma.be/oralsite/pages/What's\_the\_Score\_Publication/}
                                
                                         \item {{\it Écriture en Mésopotamie}, article wikipédia}
                                         \item {{\it Art rupestre}, article wikipédia}
                                
                                        
                                   \stopitemize
                                   
                                   
      \blank                                                                
      \subsubject { Vidéos}                            
                                   \startitemize 
                                   
                                         \item {{\it Fear \& Loathing in Las Vegas}, extrait
                                         
                                         http://www.youtube.com/watch?v=vUgs2O7Okqc}
                                         
                                         \item {Jocelyn Cottencin \& Tiago Guedes' Vocabulario: 
                                         
                                         http://www.dailymotion.com/video/x3oclq\_vocabulario-jocelyn-cottencin-feat\_creation }
                                         \item {Plotter OSP, vidéo youtube}                                
                                        
                                   \stopitemize


\stoptext
\stopcomponent