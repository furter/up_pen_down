\startcomponent *

\environment styles

\environment typo

\enableregime[utf-8] 

   \starttext
    \startalignment[right]
    
    
               \rotate[rotation=90,frame=off,offset=120pt]{\chapter {DXY's \crlf  Perambulation}}


   %titrecourant (rappel de nom_chapitre)
        \setupheader[style=\tfxx]
      \setupheadertexts [{\getmarking[ ]}]     [\tfx{} ]
      
\rotate[rotation=90,frame=off]{\tfb {1}}
\externalfigure[IMAGES/PL/photo.JPG][width=80mm]
    \page
\rotate[rotation=90,frame=off]{\tfb {2}}
\externalfigure[IMAGES/PL/plotter.jpg][width=80mm]
    \page

 \externalfigure[IMAGES/PL/roland1.jpeg][width=80mm]
  \page

 \externalfigure[IMAGES/PL/roland2.jpeg][width=80mm]
\page

 %\externalfigure[IMAGES/PL/roland3.jpeg][width=100mm]
%\page
 \externalfigure[IMAGES/PL/roland4.jpeg][width=80mm]
 
\page
\rotate[rotation=90,frame=off]{\tfb {4}}
 \externalfigure[IMAGES/PL/figure1.jpg][width= 200px]
  \setupinterlinespace[8.5pt]     
\logossssb
{Rob Mallet-Stevens et Paul Ruaud, architectes, Henri Laurens, sculpteur. \crlf Projet pour la ville de Jacques Doucet, 1924, dessin de Rob Mallet-Stevens \crlf pour l'entrée.}
 \page


 \externalfigure[IMAGES/PL/figure2.jpg][width= 200px]
 \logossssb {Le Corbusier, Urbanisation d'Hellocourt, 1935.}
  \page

 
  \externalfigure[IMAGES/PL/figure3.jpg][width= 200px]
 \logossssb{ Jacques Berce, Henri E. Ciriani, Michel Corajoud, Borja Huidobro, \crlf Georges Loiseau, Annie Tribel, Jean Tribel (AUA), \crlf Projet d'habitat modulaire, «Tétrodon», 1970-1972.}
 \page

  
  \externalfigure[IMAGES/PL/figure4.jpg][width= 200px]
\logossssb{Roland Castro, Abd-el-Krim Driss, Guy Duval, Lorenzo Maggio (avec l'aide d'Antoine Stinco), Projet pour le concours sur le terrain de l'ancienne prison de la Roquette, Paris, 1974.}
 \page


 \externalfigure[IMAGES/PL/figure6.jpg][width= 200px]
 \logossssb { Johan Otto von Spreckelsen, la «Grande Arche» de la Défense, \crlf Projet lauréat du concours international pour la Tête-Défense, 1983.}
 \page


 \externalfigure[IMAGES/PL/figure7.jpg][width= 200px]
  \logossssb { Herzog & de Meuron and Ai Weiwei, Serpentine Gallery Pavilion 2012}
  \page


 \rotate[rotation=90,frame=off]{\tfb {6}}
 \externalfigure[IMAGES/PL/ce3.jpg][width= 80mm]
 \page

 \rotate[rotation=90,frame=off]{\tfb {7}}
 \externalfigure[IMAGES/PL/roland.jpeg][width= 80mm]
 
\page 

   %titrecourant (rappel de nom_chapitre)
    \setupheader[style=\tfxx]
      \setupheadertexts [{\getmarking[chapter]}]     [\tfx{Git, ce connard} ]
 \setupinterlinespace[12pt]     

\tfa {1 — Roland}

\logosssb { Le plotter Roland étant à notre disposition, interroger l'outil nous semblait être une piste intéressante. Dans un but d'expérimentation formelle, nous sommes revenus à son époque d'utilisation en ayant pour idée de le confronter à d'autres moyens d'impressions tels que l'offset. Nous avons cherché diverses possibilités d'expérimentations formelles : \crlf
— Utiliser le plotter pour graver directement la plaque offset\crlf
— Insoler la plaque offset à partir d'un typon réalisé au plotter\crlf
— Décomposition des couches CMJN}

\blank
\tfa {{2 — Perambulation}}

\logosssb {La recherche d'un contenu à produire qui parlerait directement de l'outil nous a conduits vers la page test d'impression du plotter Roland, disponible dans la machine.
--> Réalisation d'essais et utilisation de différents outils --> rendu graphique multiple sur la même image.}

\tfa {3 — Page test}

\logosssb {Cette page révèle clairement le contexte d'utilisation, l'époque et l'usage du plotter, un outil pour les architectes servant à tracer des plans et des axonométries.}
\page
\tfa{4 — Architecture}
\logosssb {Le plotter a défini l'imagerie des projets architecturaux d'une certaine époque.
Question : Comment cette imagerie évolue-t-elle en même temps que les outils? 
Sélection d'une dizaine d'images de projets architecturaux du XX\high{e} siècle pour les tests
--> Questionner l'outil utilisé et la manière dont il détermine un visuel de présentation de projets architecturaux.}

\tfa{5 — Perambulation}

\logosssb {Après réflexion, nous nous sommes dit que nous manquions de connaissances théoriques et d'expérience en la matière pour aboutir à une réflexion pertinente sur le propos. 
Nous sommes revenus à la définition du plotter (un outil pour tracer des plans) et à sa particularité de se  déplacer à travers le plan de la feuille pour dessiner.}
\startquotation[left]
\citations {Un plotter est un périphérique d'impression informatique pour les impressions graphiques en mode trait. Depuis les années 1960, les traceurs à plume ou jet d'encre ont accompagné l'expansion de la conception assistée par ordinateur. Depuis les années 1980, ils ont généralement été remplacés par des imprimantes à jet d'encre et laser de grand format, de sorte qu’il est maintenant courant de se référer à ces imprimantes grand format comme « traceurs », même si, ces imprimantes n'utilisent pas le « tracé » comme technique d'impression.}
\stopquotation
\logossssb http://fr.wikipedia.org/wiki/Traceur_(informatique)
\page
\logosssb {Nous avons donc envisagé l'architecture comme la mise en relation d'une forme avec un espace et des corps. Pour rendre compte de cela, nous avons pensé une intervention sur le déplacement du plotter dans l'espace de la page. Nous nous sommes interrogés sur la définition d'un espace (d'une architecture) par la déambulation.} 

\tfa{6 — Brightness Tracking}

\logosssb {Pour concrétiser assez rapidement cette piste, utilisation d'un script Processing capturant grâce à la webcam la trajectoire de la lumière point par point.}

\tfa{7 — DXY Perambulation}

\logosssb {Transmission au plotter de ces coordonnées \crlf --> tracé d'un déplacement à l'échelle de la page A3 du plotter.}
\crlf
\crlf
\crlf
\crlf\crlf\crlf\crlf
 \startalignment[left]
\logosssb {A.J / D.V / G.F }

\stopalignment
\stoptext

\stopcomponent
