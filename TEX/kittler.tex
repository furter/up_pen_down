\startproduct kittler

	\starttext 
		

\title{Une histoire d'amour ratée avec une machine à écrire}
Friedrich Kittler

"Nos outils d’écriture travaillent aussi nos pensées", a écrit Nietzsche [1]. "La technologie est retranchée dans notre histoire", a dit Heidegger. Mais l’un a écrit cette phrase à propos de la machine à écrire sur une machine à écrire quand l’autre décrivait (dans un allemand ancien magnifique) l’essence de la machine à écrire. Voilà pourquoi c’est Nietzsche qui fut à l’origine de la transvaluation des valeurs avec sa phrase philosophiquement scandaleuse sur la technologie des médias, "les hommes ne sont peut-être que des machines pensantes, écrivantes et parlantes". En 1882, aux êtres humains, à leurs pensées et à la figure de l’auteur se sont substitués terme à terme les deux sexes, le texte et l’appareillage de l’écriture à l’aveugle. Premier philosophe mécanisé, Nietzsche fut aussi le dernier. Le texte dactylographié avait pour nom, si l’on en croit la peinture de Klapheck, Volonté de puissance [2].


	\stoptext
\stopproduct
