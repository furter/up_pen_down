\startproduct baudrillard

	\starttext 
		

\title{Le Xerox et l'infinity}

Jean Baudrillard

Ce qui distinguera toujours le fonctionnement de l'homme et celui des machines, même les plus intelligentes, c'est l'ivresse de fonctionner, le plaisir. Inventer des machines qui aient du plaisir, voilà qui est heureusement encore au delà des pouvoirs de l'homme. Toutes sortes de prothèses peuvent aider à son plaisir, mais il ne peut en inventer qui jouiraient à sa place. Alors qu'il en invente qui travaillent, "pensent" ou se déplacent mieux que lui ou à sa place, il n'y a pas de prothèse, technique ou médiatique, du plaisir de l'homme, du plaisir d'être homme. Il faudrait pour celà que les machines aient une idée de l'homme, qu'elles puissent inventer l'homme, mais pour elles il est déjà trop tard, c'est lui qui les a inventées. C'est pourquoi l'homme peut excéder ce qu'il est , alors que les machines n'excéderont jamais ce qu'elles sont. Les plus intelligentes ne sont exactement que ce qu'elles sont, sauf peut-être dans l'accident et la défaillance, qu'on peut toujours leur imputer comme un désir obscur. Elles n'ont pas ce surcroit ironique de fonctionnement, cet excès de fonctionnement en quoi consistent le plaisir ou la souffrance, par où les hommes s'éloignent de leur définition et se rapprochent de leur fin. Hélas pour elle, jamais une machine n'excède sa propre opération, ce qui peut-être explique la mélancolie profonde des computers... toutes les machines sont célibataires. (pourtant la récente irruption des virus électroniques offre une anomalie remarquable : on dirait qu'il y a un malin plaisir des machines à amplifier, voire à produire des effets pervers, à excéder leur finalité par leur propre opération. Il y a là une péripétie ironique et passionnante. Il se peut que l'intelligence artificielle se parodie elle même dans cette pathologie virale, inaugurant par là une sorte d'intelligence véritable.) 



	\stoptext
\stopproduct

