\startcomponent 
\enableregime[utf-8] 

 \usemodule[simplefonts]

\definesimplefont[logo][logo][size=9pt] 


\setupbodyfont[logo, 12pt]

\starttext

\title {
DXY's perambulation
}
\logo
Photo du plotter
1 - Le plotter Roland étant à notre disposition, interroger l'outil nous semblait être une piste intéressante. Dans un but d'expérimentation formelle, nous sommes revenu à son époque d'utilisation en ayant pour idée de le confronter à d'autres moyens d'impressions tels que l'offset. Nous avons cherché diverses possibilités d'expérimentations formelles, comme l'utilisation le plottage direct sur la plaque offset, l'insolation de la plaque offset à partir d'un typon ou la décomposition d'une image CMJN.

 Photo des commandes du plotter
2 - La recherche d'un contenu à produire qui parlerait directement de l'outil nous a conduit vers la page test d'impression du plotter Roland, disponible dans la machine. 

Scanns des page tests
3 - Cette page révèle clairement le contexte d'utilisation l'époque et l'usage du Plotter, un outil pour les architectes servant à tracer des plans et des axonométries. 

Images d'architecture
4 - Dans un but d'expérimentation du plotter et au vu de l'apparition marqué d'une époque, nous avons chercher à voir comment pourrait réagir des plans d'architecture et des axonométries à travers les époques, et plus particulièrement le XXe siècle. 
Nous avons  sélectionné une dizaine d'images pour les tests, et voir ainsi comment l'outil utilisé détermine un visuel de présentation de projet architecturaux.

5 - Après réflexion, nous nous sommes dit que nous manquions de connaissances théoriques et d'expérience en la matière pour aboutir à une réflexion pertinente sur le propos.
Nous sommes revenu à la définition du plotter (un outil pour tracer des plans) et à sa particularité de se  déplacer à travers le plan de la feuille pour dessiner. Nous avons donc envisagé l'architecture comme la mise en relation d'une forme avec un espace et des corps. Pour rendre compte de cela, nous avons pensé une intervention sur le déplacement du plotter dans l'espace de la page. Nous nous sommes interrogé sur la définition d'un espace (d'une architecture) par la déambulation. L'idée du projet était de tracker la position d'une personne dans un espace à l'aide de son téléphone. 

6 - Pour concrétiser assez rapidement cette piste, nous avons utilisé un code Processing capturant à la webcam la trajectoire de la lumière point par point. 

7 - Transmises au plotter, ces coordonnées ont donné lieu au tracé d'un déplacement à l'échelle de la page A3 du plotter. 


\stoptext