\startcomponent colophon
\title {Colophon}
\starttext
"Les oiseaux migrateurs doivent parcourir de très longues distances, dans des
conditions parfois difficiles. Ainsi, il est important pour eux d'optimiser leur
déplacement en termes d'énergie dépensée. Les oies sauvages adoptent des
formations en V qui leur permettent d'étendre leur distance de vol de près de
70\%, car chaque oiseau prend l'aspiration de son prédécesseur, comme le font
les cyclistes.
Le prix à payer est une perte en vitesse, puisqu'un individu seul vole en
moyenne 24\% plus vite qu'une volée."
http://fr.wikipedia.org/wiki/Intelligence_collective

"La sociocratie est un mode de prise de décision et de gouvernance qui permet à
une organisation de se comporter comme un organisme vivant et de
s'auto-organiser. L'objectif premier est de développer la co-responsabilisation
des acteurs et de mettre le pouvoir de l'intelligence collective au service du
succès de l'organisation. La méthode sociocratique est fondée sur le concept
plus aucune objection argumentée d'aucune personne."
http://fr.wikipedia.org/wiki/Sociocratie

Environ à la médiane du workshop qui a produit ce livret, une session autour
d'une table ronde s'est déroulé sous un mode relativement sociocratique et a
défini les grandes lignes de l'organisation à mettre en place pour terminer le
document dans les délais. On ne saura sans doute jamais si cette expérience
particulière a permit certains huilages de roue de conception ou si le temps
nécessaire à l'expérimenter aurait pu servir avantageusement à d'autres tâches.
Mais elle a en tout cas été l'occasion d'articuler les mots alibi, autorisation,
contrainte, critique, signer, sincère, entre autres et dans l'ordre
alphabétique.

Up Pen Down - Workshop - septembre 2012 à janvier 2013
Atelier des étudiants en 4e année Design Graphique à Valence, avec Open Source
Publishing (Alexandra Rio, Alice Jauneau, Anaïs Alauzen, Camille Chatelaine,
David Vallance, Éléonore Jasseny, Gwenaël Fradin, Jordane Cals, Lorène Ceccon,
Margot Baran et Vincent Duché avec Alexandre Leray, Eric Schrijver, Femke
Snelting, Gijs De Heij, John Haltiwanger, Ludi Loiseau, Pierre Huyghebaert,
Pierre Marchand, Stéphanie Vilayphiou)

Le texte d'intention initial s'énonçait comme suit :
Notre intervention se place autour du “trait”, dans le sens de “chemin” (path en
anglais) par opposition et en dialogue avec la notion de forme. Point de départ
offrant plusieurs perspectives - celles du dessin, de la typographie, de la
cartographie - ce “chemin” pose aussi la question des outils - en particulier
numériques - et de leur relation avec le langage visuel.
En tant que constructions techniques, intellectuelles et culturelles, les
logiciels incarnent des conceptions spécifiques des objets qu’ils manipulent
(ici, le trait et la forme). Le format PostScript, par exemple, décrit les
glyphes par leur contour plutôt que par leur squelette, amenant ainsi à une
conception spécifique de la lettre, mais il existe d'autres formats de
description moins connus qui prennent l’approche inverse.
Le logiciel libre, par la disponibilité du code mais surtout par l'ouverture des
discussions relatives à l’élaboration des outils (par exemple par le biais des
listes de discussion, ou de l'utilisation de logiciels de gestion des versions),
permet de mieux saisir les implications des outils dans le travail du designer."

Une large série d'outils a été utilisés, dont les plus mémorables sont sans
doute :


\startitemize 

\item {Git est un logiciel de gestion de versions décentralisé. Nous travaillons
sur des centaines de fichiers graphiques et surtout sur des dizaines de fichiers
textes qui mêlent contenu et code. Certains d'entre nous travaillent sur le même
fichier en même temps, à des endroits différents. Le genre de situation qui mêne
à des catastrophes dans le quotidien des studios de graphisme. Le changement de
paradigme du versionnage de fichier permet non seulement de les éviter, mais
surtout de construire une pratique du travail en groupe auto-instruit et qui
construit sa propre histoire en parallèle par le biais des commentaires
nécessaires à insérer à chaque publication de version. Le prix à payer est
l'apprentissage d'un rythme de commandes qui régule le rapport entre la version
intime des documents, et celle plus collective. Ce rythme et son vocabulaire,
fait de dépot, de commettre et de pull, demande un peu de patience et d'avancer
avec confiance à tâtons dans le brouillard avant de montrer ses premiers signes
rassurants, et même efficaces.}

\item {Github est la plateforme web qui permet d'éviter d'utiliser son propre
serveur pour offrir des services Git. L'inconvénient de la dépendance est jugé
ici raisonnable vu le caractère relativement ponctuel du workshop.}
 
\item {Vim et Vi, l'éditeur de texte vénérable qui s'ouvre parfois pour qu'on y
insère les commentaires que l'on commet.}

\item {Virtualbox pour faire essayer Linux sur son ordinateur familier
relativement aisément.}

\item {Turtle, de quoi faire du Logo et back to the future.}

\item {ConTeXt est un outil de mise en page et de composition typographique.
ConTeXt est une version aidante et problématique dans le même temps du logiciel
grand-père TeX. Le dernier workshop traverse ConTeXt en oblique, frontalement et
de côté. Un peu à la manière dont la culture et la conception particulière de la
pratique de la mise en page cosmétique de son créateur nous a électrisé et
embrumé.}

\item {FontForge est le principal et archétypal éditeur libre de fontes. On
l'aime comme une.}

\item {Inkscape, éditeur vectoriel pico bello.}

\item {Gimp, éditeur d'images quand besoin.}

\item {TexShop, parce que la plupart d'entre nous édite ses fichiers TeX sur
Mac. Mais la compilation intégrée, entre autres, est fort problématique.}

\item {Chiplotle, pour lancer le tracage par notre Roland, plotter table
traçante.} 

\stopitemize

\stoptext
\stopcomponent
