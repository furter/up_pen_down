\startcomponent intro
\environment styles
\environment typo
\enableregime[utf-8] 
\title {Git, ce\crlf connard}
\subsubject {Introduction}
\starttext
 \startalignment[right]
Git  est un logiciel de gestion de versions décentralisé, créé par Linus Torvalds, auteur du noyau GNU/Linux, et distribué selon les termes de la licence publique générale GNU version 2.
\blank
\startquotation[left]
\citations{
Quand on lui a demandé pourquoi il avait appelé son logiciel "git", qui est à peu près l'équivalent de "personne pourrie" en  argot britannique, Linus Torvalds a répondu "je ne suis qu'un sale égocentrique, donc j'appelle tous mes projets d'après ma propre personne. D'abord Linux, puis Git.}
\stopquotation
Contributeurs de Wikipédia, "Git,"  Wikipédia, l'encyclopédie libre, 
http://fr.wikipedia.org/w/index.php?title=Git/&oldid=87393315 (Page consultée le janvier 11, 2013).
\blank
Nous avons travaillé autour de la notion de "stroke", du trait. Nous avons expérimenté des choses dans des zones brumeuses mais dans lesquelles une lumière nous attirait. De la performance, du dessin de caractère, du hardware hacking, de la mise en page, de la cuisine, de la sociocratie, de chorégraphies à base d'éponges, ou encore de gros risques vestimentaires.
\blank
Nous avons exploré différentes directions, toujours accompagnés par le fameux Git. Ce qui nous a parfois mené à de violents maux d'estomac car confrontés à des erreurs intempestives, des fichiers à "merger", des conflits de caractères, ce qui pouvait donner lieu à un rejet total des méthodes et outils de travail du libre. "Chacun sa merde" émergeait alors comme un divin {\it statement}. 
\blank
Finalement, le libre nous a eus. Et cet objet rend compte de cette relation parfois houleuse, mais finalement excitante dans son infinie découverte. Et aujourd'hui, peut-être avouerons-nous à demi 
mot, qu'il nous a séduits…
\stopalignment
\stoptext
\stopcomponent
