\starttext

DXY's perambulation

Multiplicité des résultats du plotter

Trace/trame/texture
Couches
Typon -> transparence -> trace -> specificité
Résistance/confrontation de l'outil plotter aux outils de reproduction de son époque
Insolation avec un rétroprojecteur plaque offset/sérigraphie
Document de test du plotter -> apparition d'une fonction, d'une époque.
Confrontation de deux modes de reproductions, le plotter qui correspond aux années 1960 et à un usage spécifique et l'offset qui correspond aux attentes de productions actuelles.
1- Proposition d'impression de page test offset en utilisant des trames réalisées avec le plotter.
> Décomposition des couche cmjn, apparition de l'outils plotter au travers de l'offset
2- Plottage direct sur la plaque
> décalage production unique/en série
Images:
- XVIIè siècle plan, coupe et vue des façades de palais ou d'hôtels particuliers. Architecture et fortifications. 
- 
Le plotter est un outil pour tracer des plans. Il a la particularité de se déplacer à travers le plan de la feuille pour dessiner. Nous envisageons l'architecture comme la mise en relation d'une forme avec un espace et des corps. Pour rendre compte de cela dans notre projet nous pensons intervenir sur le déplacement du plotter dans l'espace de la page. Nous nous interrogeons désormais sur la définition d'un espace (d'une architecture) par la déambulation.
Direction 1.0:

Mettre en place un système simple de tracking pour pouvoir tester assez rapidement l'impression de ces mouvements dans l'espace avec le plotter.

L'idée était d'abord d'utiliser les fonctions de nos smartphone, mais en vue du temps nous allons tester un tracking lumineux avec processing.

Code processing utilisé:

/**
 * Brightness Tracking 
 * by Golan Levin. 
 * 
 * Tracks the brightest pixel in a live video signal. 
 */
 
 \externalfigure[ce3.png]

\stoptext