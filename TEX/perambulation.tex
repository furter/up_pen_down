
\enableregime[utf-8] 
\starttext

\title {
DXY's perambulation
}

\subject {Day 1 — Project 1 / Plotter Specificities}

Trace/trame/texture
Couches
Typon -> transparence -> trace -> specificité
Résistance/confrontation de l'outil plotter aux outils de reproduction de son époque
Insolation avec un rétroprojecteur plaque offset/sérigraphie
Document de test du plotter -> apparition d'une fonction, d'une époque.
Confrontation de deux modes de reproductions, le plotter qui correspond aux années 1960 et à un usage spécifique et l'offset qui correspond aux attentes de productions actuelles.
1- Proposition d'impression de page test offset en utilisant des trames réalisées avec le plotter.
> Décomposition des couche cmjn, apparition de l'outils plotter au travers de l'offset
2- Plottage direct sur la plaque
> décalage production unique/en série


Page test DXY Roland


 \externalfigure[IMAGES/PL/roland1.jpeg][width=200px]


 \externalfigure[IMAGES/PL/roland2.jpeg][width=200px]


 \externalfigure[IMAGES/PL/roland3.jpeg][width=200px]


 \externalfigure[IMAGES/PL/roland4.jpeg][width=200px]
 
 

\subject {Day 2 — Project 2 / Page test}

 
 Images
- XVIIè siècle plan, coupe et vue des façades de palais ou d'hôtels particuliers. Architecture et fortifications. 
- 
-
-
-
-
-
-

\subject {Day 3 — Project 3}

DXY's perambulation

Le plotter est un outil pour tracer des plans. Il a la particularité de se déplacer à travers le plan de la feuille pour dessiner. Nous envisageons l'architecture comme la mise en relation d'une forme avec un espace et des corps. Pour rendre compte de cela dans notre projet nous pensons intervenir sur le déplacement du plotter dans l'espace de la page. Nous nous interrogeons désormais sur la définition d'un espace (d'une architecture) par la déambulation.

Direction 1.0:

Mettre en place un système simple de tracking pour pouvoir tester assez rapidement l'impression de ces mouvements dans l'espace avec le plotter.

L'idée était d'abord d'utiliser les fonctions de nos smartphone, mais en vue du temps nous allons tester un tracking lumineux avec processing.



DXYRepresentation
Interested in tracking ourselves in the space, and outputting that directly to the plotter
we didn't have much time. So we used a script which tracks brightness.
using Golan Levin's Brightness Tracking
moving the computer, not the light
points represent the brightest points in the space. Program has a 'ticker', you can probably change timing.
idea: fix a webcam, to put a light on our heads, track trajectories
relate movement of the plotter to movement of the plotter
coordinates are connected
take the information, use it as an input. Would like to do this live
Each ten seconds the program combines the points into a ' path' and sends it to the plotter
OSP: setup to track people with light ... why? can you be more ambitious with this? there is maybe more interesting ways to do it?
It wasn't the idea to track with a cam / light / movement. We wanted to track with a phone (GPS?)
We capture movement in space.
cartography is a large field ... you will cross interesting questions: the reference systems you will cross. translating from xy (plan/blueprint/map) to a map (mercator, ...)
a shift of scale ... depending of the difference between scales
3 different reference systems that overlap, overlay, interfere
OSP does not want you to miss the point?
technical problams are cultural
if we want to realise the project ... we need a developer to do this project?
change your way of working means your project changes (every technology has its own meaning)
don't forget where you are : )

\stoptext