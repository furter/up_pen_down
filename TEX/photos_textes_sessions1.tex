\startcomponent *
\environment styles
\environment typo
\enableregime[utf-8] 

\starttext
\title {Turtle graphic} 

 The turtle has three attributes:\blank
	1.	a location\blank
	2.	an orientation\blank
	3.	a pen, itself having attributes such as color, width, and up versus down.\blank
The turtle moves with commands that are relative to its own position, such as "move forward 10 spaces" and "turn left 90 degrees". The pen carried by the turtle can also be controlled, by enabling it, setting its color, or setting its width. A student could understand (and predict and reason about) the turtle's motion by imagining what they would do if they were the turtle. Seymour Papert called this "body syntonic" reasoning.
From these building blocks one can build more complex shapes like squares, triangles, circles and other composite figures. Combined with control flow, procedures, and recursion, the idea of turtle graphics is also useful in a Lindenmayer system for generating fractals.
Turtle geometry is also sometimes used in graphics environments as an alternative to a strictly coordinate-addressed graphics system.
Turtle graphics were added to the Logo programming language by Seymour Papert in the late 60s to support Papert's version of the turtle robot, a simple robot controlled from the user's workstation that is designed to carry out the drawing functions assigned to it using a small retractable pen set into or attached to the robot's body. Turtle geometry works somewhat differently from (x,y) addressed Cartesian geometry, being primarily vector-based (i.e. relative direction and distance from a starting point) in comparison to coordinate-addressed systems such as PostScript. As a practical matter, the use of turtle geometry instead of a more traditional model mimics the actual movement logic of the turtle robot. The turtle is traditionally and most often represented pictorially either as a triangle or a turtle icon (though it can be represented by any icon).
Papert's daughter, Artemis, has been using turtle graphics to explore the relationship between art and algorithm.
Today, Turtle graphics implementations are available for all major desktop and mobile platforms.\blank

\useURL[lien : http://en.wikipedia.org/wiki/Turtle_graphics]

 \page
 
 %\setupcolumns[n=2, ntop=4]

     
  \externalfigure [../IMAGES/ARC/Turtle/Num14.jpg] [width=6cm] \blank
 \externalfigure [../IMAGES/ARC/Turtle/Num17.jpg] [width=10cm]\blank 
 Article Pyramide Diapason Roue Crantée n°3 "Seymour et la Turtle"
 

 \page

  %\setupcolumns[n=2, ntop=4]
     \startcolumns[n=2]


\title {Turtle\blank Expérience}

\blank
 C'est autour du concept de la Turtle que nous avons basé notre démarche de dessin durant le workshop. La Turtle, née dans les années 70, inventé par Hal Abelson et Andrea Disessa, propose aux jeunes élèves d'appréhender et d'explorer les propriétés visuelles mathématiques à travers un langage très simple de programmation. A l'aide de ce langage, il s'agit de manoeuvrer une "tortue" pour qu'elle trace lignes et courbes.

Chaque ligne dessinée est  déterminée par l'angle directionnel ainsi que par un pas d'avancé déterminant la longueur du tracé. Il faut ainsi concevoir la tortue comme une entité se déplaçant sur la feuille/page virtuelle. On considère que l'on donne des indications de déplacement à cette entité qui par son parcours sur la page dessine une forme.       
Cherchant dans notre projet à explorer le code comme une partition, on se rapproche de la partition de danse qui semble entrer en résonnance. 
Comment le code devient-il l'outils de création dans l'espace ?
L'outil de la Turtle se révèle être un biais très pédagogique grâce à sa simplicité pour s'approprier et explorer cette question en jouant avec le corps dans l'espace.
\blank

  \externalfigure [../IMAGES/ARC/photos/Image 5.png]\blank
         \externalfigure [../IMAGES/ARC/photos/Image 8.png]\blank
         \externalfigure [../IMAGES/ARC/photos/Image 9.png]\blank
       \externalfigure [../IMAGES/ARC/photos/Image 11.png]\blank
         1ère session. Appropriation des outils dans l'espace. \blank
         
\page
 La mise en espace de cette technique exige qu'il y ait d'une part un instructeur, c'est à dire un interlocuteur qui donne les instructions suivant le langage de Turlte, à haute voix et d'une autre part une Turlte, soit la personne qui reçoit et exécute les instructions données.

La Turtle n'écrit pas avec ses pieds, elle est armée d'un pinceau. Nous avons donc fabriqué un certain nombre de pinceaux à partir d'éponge découpé (= plume du pinceau) et de latte de bois (= manche du pinceau). Cependant, ce sont bien ses pieds qui servent d'indice de mesure pour les déplacements.


Très vite, les Turtles envahissent le sol de traces lisibles ou... non. L'expérimentation est riche en embuche. Le code pose parfois question, à la fois dans la logique de déplacement de la Turtle, comment dans la création du plein et du délié. 

En effet, la Turtle seule semble avoir du mal à proposer une plume avec un délié efficient. Par rapport à la taille des lettres dessinées, les plumes fabriquées ne sont pas assez grosse. Alors nous vient l'idée de diriger deux tortues sur un même tracé. Ainsi selon l'écart choisi entre les tortues c'est la taille du fût qui varie. De la même manière, si une tortue s'avance d'un pas avant les instructions, alors c'est l'angle de la plume qui varie. 

À travers notre expérimentation du langage, c'est une version augmentée de la Turlte que l'on met ainsi à l'exercice.





         \externalfigure [../IMAGES/ARC/photos/Image 7.png]\blank       
         \externalfigure [../IMAGES/ARC/photos/Image 10.png]\blank
         \externalfigure [../IMAGES/ARC/photos/Image 11.png]\blank
         \externalfigure [../IMAGES/ARC/photos/Image 13.png]\blank
         1ère session. Appropriation des outils dans l'espace. \blank
         \externalfigure [../IMAGES/ARC/photos/Image 14.png]\blank
         \externalfigure [../IMAGES/ARC/photos/Image 15.png]\blank
         \externalfigure [../IMAGES/ARC/photos/Image 16.png]\blank          
         \externalfigure [../IMAGES/ARC/photos/Image 17.png]\blank
         \externalfigure [../IMAGES/ARC/photos/Image 18.png]\blank
         \externalfigure [../IMAGES/ARC/photos/Image 19.png]\blank
         \externalfigure [../IMAGES/ARC/photos/Image 21.png]\blank
         \externalfigure [../IMAGES/ARC/photos/Image 22.png]\blank
                  Repas collectif. Essais des outils en extérieur.\blank         
         \externalfigure [../IMAGES/ARC/photos/Image 23.png]\blank
         \externalfigure [../IMAGES/ARC/photos/Image 24.png]\blank
         \externalfigure [../IMAGES/ARC/photos/Image 25.png]\blank
         \externalfigure [../IMAGES/ARC/photos/Image 26.png]\blank
         \externalfigure [../IMAGES/ARC/photos/Image 27.png]\blank
         \externalfigure [../IMAGES/ARC/photos/Image 2.png]\blank
         \externalfigure [../IMAGES/ARC/photos/Image 4.png]  \blank

         \externalfigure [../IMAGES/ARC/photos/Image 3.png]\blank
          \externalfigure [../IMAGES/ARC/photos/Image 28.png]    \blank
          Restitution publique du workshop. Performance dans le Hall de l'école.
       
              
\page  

\page
         \externalfigure [../IMAGES/ARC/photos/Tentative restitution 1.jpg]  [width=10cm]  \blank
                           Tentative de première restitution entre les deux sessions.\page
         

\title {No-feature}

No Feature est une invitation à développer un espace de travail collectif autour de la typographie dans le cadre de la section quatrième année design graphique, et plus largement au sein de l'école supérieure d'art et de design de Valence. Le projet vise à rassembler des énergies autour d'objets typographiques, quel que soit leur avancement. Les typographies sont partagées de préférence au format UFO, et publiés sous licence libre, invitant les utilisateurs à modifier ou enrichir les projets. Cet espace est ouvert à tout les étudiants de l'école, aux anciens étudiants ainsi qu'aux professeurs.
Les fichiers de travails sont rassemblés dans un dépôt git. Le site github fonctionne comme hébergeur. Un mode d’emploi pour collaborer en utilisant ces outils est à venir.\blank

          \externalfigure [../IMAGES/ARC/photos/P1070575.JPG] [width=4cm]\blank
         \externalfigure [../IMAGES/ARC/photos/P1070576.JPG] [width=4cm]
         \externalfigure [../IMAGES/ARC/photos/P1070579.JPG] [width=4cm]    \blank          
         \externalfigure [../IMAGES/ARC/photos/P1070582.JPG] [width=4cm]\blank         
         \externalfigure [../IMAGES/ARC/photos/P1070592.JPG] [width=4cm] \blank        
         \externalfigure [../IMAGES/ARC/photos/P1070595.JPG] [width=4cm]    \blank
         
         \page
         
         \stopcolumns
         
            \externalfigure [../IMAGES/ARC/photos/04.png][width=8cm]\blank
             \externalfigure [../IMAGES/ARC/photos/05.png][width=8cm]
          Travail de la typographie par Alice et Gwenaël.
          Page de droite, blog en ligne de collecte des pistes de travail pour la 3éme session.
             
        \blank

             
             
         
         
         
           
 \page
         \externalfigure [../IMAGES/ARC/photos/http---uppendown.hotglue.me-Pistes de recherche du projet en relation au trait .jpg][width=20cm]

         \externalfigure[../IMAGES/ARC/photos/http---no-feature.github.com-.jpg][width=20cm]
        \blank

     Page de la fonderie de No-Feature qui réunit toutes les productions typographiques.
        \useURL [uppendown.hotglue.me]
        \useURL [no-feature.github.com] 

 







      
         
\stoptext 
\stopcomponent 
