\startcomponent dr
\starttext
\part {Dr._Karl_Hess_Keffer}


\externalfigure[../IMAGES/DR/Experience1.jpg][width=17cm]

\externalfigure[../IMAGES/DR/experience4.jpg][width=17cm]

\externalfigure[../IMAGES/DR/experience5.jpg][width=17cm]

\externalfigure[../IMAGES/DR/xp3.jpg][width=17cm]

\externalfigure[../IMAGES/DR/xp6.jpg][width=17cm]

\externalfigure[../IMAGES/DR/xp2.jpg][width=17cm]



\externalfigure[../IMAGES/DR/outils_experience1.jpg][width=15cm]

\externalfigure[../IMAGES/DR/outils_experience6.jpg][width=15cm]



\externalfigure[../IMAGES/DR/scan1.jpg]

\externalfigure[../IMAGES/DR/scan2.jpg]

\externalfigure[../IMAGES/DR/scan3.jpg]

\externalfigure[../IMAGES/DR/scan4.jpg]

\externalfigure[../IMAGES/DR/scan5.jpg]

\externalfigure[../IMAGES/DR/scan6.jpg]




Étude de la pensée collective par le Dr Karl Hess Keffer

Restitution d'une archive des expériences d'outils réalisées par le Dr. Hess Keffer, neuro-scientifique.
Ces outils tentent de trouver un lieu d'expression de la pensée collective. 

Dr. Karl Hess Keffer a fait de nombreuses recherches sur l'activité cérébrale. 
Ces expériences mettent en jeu différents aspects de la collaboration. 

Cette recherche utopique n'a jamais trouvé de réponse ultime, car chaque outil met en lumière une faille: l'impossibilité d'un systématique consensus commun. 

Après 35 ans de recherches, le Dr Karl Hess Keffer abandonna son étude vaine. Actuellement, il travaille sur l'instinct grégaire des animaux. 

\stoptext
\stopcomponent