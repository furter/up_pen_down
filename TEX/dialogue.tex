\startproduct dialogue

	\starttext 
		\part Prémices du projet
		

\title{Dialogue}

 À partir d'une réflexion sur les outils et les médias apportÉe par OSP, nous avons rencontré plusieurs lectures. Le geste d'écrire, de Vilém Flusser, a été pour nous le moteur de notre questionnement sur la relation entre l'homme et la machine. Après un exercice de mise en commun des références, deux artistes ont, pour nous fait sens. Puisant dans le travail sur les frontières de Francis Alÿs (Sometimes doing something poetic can become political and sometimes doing something political can become poetic) et le projet mesuRages d'Orlan, nous avons voulu mettre en avant l'utilisation du corps comme comme outils. De là découle pour nous plusieurs questions élémentaires, comme qu'est-ce qu'une machine ? Quelle est la différence entre nous et la machine ? Ainsi que Est-ce que la sensibilité a une place légitime dans un rapport homme-machine ?


\title{Le geste d'écrire}

Vilém Flusser

La structure du geste est linéaire. Mais il s'agit d'une linéarité spécifique. On commence, en thèse ; par le coin supérieur à gauche de la surface, on fait une ligne jusqu'à ce qu'on arrive au coin supérieur à droite, on saute à gauche pour recommencer le geste un peu plus en bas, et on répète ce mouvement jusqu'au coin inférieur droit. Il s'agit d'une structure apparemment accidentelle : elle à été imposée sur le geste par les accidents de notre histoire. Elle pourrait être différente, et en effet, elle l'est dans d'autres civilisations. Néanmoins, cette structure là, qui est le résultat d'accidents méprisables comme l'est la qualité de la boue en Mésopotamie. ordonne toute une dimension de notre être-dans-monde : elle ordonne nos pensée linéaires, logiques, historiques, scientifiques. Car nous sommes programmés pour ce type de pensées par notre écriture, et, inversement, ces pensés sont programmés pour être écrit selon la structure que je viens de décrire. Le moindre changement dans cette structure changerais, sans doute, ce type de pensée. Mais, bien sûr, l'inverse est aussi vrai : tout changement structurel dans nos pensées implique un changement dans la structure de l'écriture. Peut-être est-ce en train d'arriver à présent.



\title{Le Xerox et l'infinity}

Jean Baudrillard

Ce qui distinguera toujours le fonctionnement de l'homme et celui des machines, même les plus intelligentes, c'est l'ivresse de fonctionner, le plaisir. Inventer des machines qui aient du plaisir, voilà qui est heureusement encore au delà des pouvoirs de l'homme. Toutes sortes de prothèses peuvent aider à son plaisir, mais il ne peut en inventer qui jouiraient à sa place. Alors qu'il en invente qui travaillent, "pensent" ou se déplacent mieux que lui ou à sa place, il n'y a pas de prothèse, technique ou médiatique, du plaisir de l'homme, du plaisir d'être homme. Il faudrait pour celà que les machines aient une idée de l'homme, qu'elles puissent inventer l'homme, mais pour elles il est déjà trop tard, c'est lui qui les a inventées. C'est pourquoi l'homme peut excéder ce qu'il est , alors que les machines n'excéderont jamais ce qu'elles sont. Les plus intelligentes ne sont exactement que ce qu'elles sont, sauf peut-être dans l'accident et la défaillance, qu'on peut toujours leur imputer comme un désir obscur. Elles n'ont pas ce surcroit ironique de fonctionnement, cet excès de fonctionnement en quoi consistent le plaisir ou la souffrance, par où les hommes s'éloignent de leur définition et se rapprochent de leur fin. Hélas pour elle, jamais une machine n'excède sa propre opération, ce qui peut-être explique la mélancolie profonde des computers... toutes les machines sont célibataires. (pourtant la récente irruption des virus électroniques offre une anomalie remarquable : on dirait qu'il y a un malin plaisir des machines à amplifier, voire à produire des effets pervers, à excéder leur finalité par leur propre opération. Il y a là une péripétie ironique et passionnante. Il se peut que l'intelligence artificielle se parodie elle même dans cette pathologie virale, inaugurant par là une sorte d'intelligence véritable.) 


\title{Une histoire d'amour ratée avec une machine à écrire}
Friedrich Kittler

"Nos outils d’écriture travaillent aussi nos pensées", a écrit Nietzsche [1]. "La technologie est retranchée dans notre histoire", a dit Heidegger. Mais l’un a écrit cette phrase à propos de la machine à écrire sur une machine à écrire quand l’autre décrivait (dans un allemand ancien magnifique) l’essence de la machine à écrire. Voilà pourquoi c’est Nietzsche qui fut à l’origine de la transvaluation des valeurs avec sa phrase philosophiquement scandaleuse sur la technologie des médias, "les hommes ne sont peut-être que des machines pensantes, écrivantes et parlantes". En 1882, aux êtres humains, à leurs pensées et à la figure de l’auteur se sont substitués terme à terme les deux sexes, le texte et l’appareillage de l’écriture à l’aveugle. Premier philosophe mécanisé, Nietzsche fut aussi le dernier. Le texte dactylographié avait pour nom, si l’on en croit la peinture de Klapheck, Volonté de puissance.










		\stoppart
	\stoptext
\stopproduct
