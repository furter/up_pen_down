\startproduct flusser

	\starttext 
		

\title{Le geste d'écrire}


Vilém Flusser

La structure du geste est linéaire. Mais il s'agit d'une linéarité spécifique. On commence, en thèse ; par le coin supérieur à gauche de la surface, on fait une ligne jusqu'à ce qu'on arrive au coin supérieur à droite, on saute à gauche pour recommencer le geste un peu plus en bas, et on répète ce mouvement jusqu'au coin inférieur droit. Il s'agit d'une structure apparemment accidentelle : elle à été imposée sur le geste par les accidents de notre histoire. Elle pourrait être différente, et en effet, elle l'est dans d'autres civilisations. Néanmoins, cette structure là, qui est le résultat d'accidents méprisables comme l'est la qualité de la boue en Mésopotamie. ordonne toute une dimension de notre être-dans-monde : elle ordonne nos pensée linéaires, logiques, historiques, scientifiques. Car nous sommes programmés pour ce type de pensées par notre écriture, et, inversement, ces pensés sont programmés pour être écrit selon la structure que je viens de décrire. Le moindre changement dans cette structure changerais, sans doute, ce type de pensée. Mais, bien sûr, l'inverse est aussi vrai : tout changement structurel dans nos pensées implique un changement dans la structure de l'écriture. Peut-être est-ce en train d'arriver à présent.




	\stoptext
\stopproduct

