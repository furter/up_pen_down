\startcomponent *
\environment styles
\environment typo
\enableregime[utf-8]
\starttext
\startalignment[right]		
\subsubject {Sur le langage}

\subsubsubject {Début d'un langage}
		
Notre langage se développe en fonction de l'outil que nous utilisons et de l'objectif à atteindre, et si « nos outils d'écritures travaillent aussi nos pensées \footnote{Nietzsche} », nous pouvons aussi dire que nos outils déterminent notre langage. Au cours de nos expérimentations, notre condition humaine provoque un langage en écho avec celle-ci : des phrases spontanées, des intonations, des mouvements, des adaptations, à contrario d'un langage plus cartésien, dénué de sensibilité et ne laissant pas forcément place à l'erreur. Ce même langage évolue au fil de l'expérience, il se transforme, s'enrichit et n'est pas figé. Notre langage comporte des accidents et le résultat nos expériences. Dépendant de l'attitude de l'autre, nous concluons que notre langage est le fruit d'une corrélation entre deux personnes. Il s'agit d'un véritable dialogue, surgissant naturellement de nos tentatives, qui s'anime et se précise au cours de la pratique. 



\subsubsubject{Observations}

\subsubsubsubsubject {Introduction d'un langage \crlf intuitif/inconscient/sensible \crlf dans notre processus d'écriture}

Les actions \blank
Celui qui tient le stylo = + passif \crlf
- action de pression/souplesse du stylo = plus fort /moins fort/moins raide…\crlf
- serrage du stylo = + ou -\crlf
- concentration = diminution et augmentation \crlf
- relâchement du corps = pauses obligatoires \crlf
- fermer les yeux \crlf
Celui qui tient la table = + dans l'action \crlf
- vitesse dans le maniement de l'outil = plus vite/moins vite \crlf
- concentration = diminution et augmentation/concentration extrême \crlf
- force = augmentation / diminution \crlf
- relâchement du corps = pauses obligatoires \crlf
- éviter le corps de l'autre = redéfinir l'espace autour de soi \crlf
- lier les lettres par le bas ou par le haut \crlf

\subsubsubsubsubject{Le langage de la table qui roule}

« Ferme les yeux ! » \crlf
« Concentre-toi ! »\crlf
« Serre-moins le stylo ! »\crlf
« Attention à ton ventre ! » « Je vais tourner »\crlf
« J'ai mal au bras »\crlf
« On fait une pause ? »\crlf
« J'en peux plus ! »\crlf
« Pfiouuuu »\crlf
« C'est lourd à la fin… »\crlf



\subsubsubsubsubject{Langage de la table qui tourne}

« Ferme les yeux ! »\crlf
« Fais des mouvements plus longs ! »\crlf
« Fais des mouvements plus courts ! »\crlf
\stopalignment



\stoptext
\stopcomponent

