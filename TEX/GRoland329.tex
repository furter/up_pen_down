\startcomponent *
\environment styles

\enableregime[utf-8] 
\starttext 
\part {expérimentations_autour_de_lechelle}


\title   \ss EXPÉRIMENTATIONS AUTOUR DE L'ÉCHELLE

{ \it. Alexandra . Anaïs . Jordane .}  \blank \blank  
Le principe de grille définit un espace dans lequel se construit la lettre, la page, le chapitre, le chemin de fer.\blank
En s'appuyant sur des grilles créées avec "nos petits robots qui dessinent" nous voulions essayer de mettre en avant les structures de ces éléments constitutifs du livre à travers ces différentes échelles. Fractalité du livre ? Qu'est-ce qui définit l'unité ? Y'a-t-il une unité pour chaque changement d'échelle ? 
Les formes des grilles créés par les "petits robots qui dessinent" ne dépendent pas de nous mais de nombreux facteurs liés au contexte dans lequel dessinent les robots. En partant de leurs accidents de parcours, des formes non maitrisées qui naissent, on obtient des grilles que nous n'aurions pas pas construites de la même manière sans ces robots. Nous posons donc ces grilles comme une contrainte qui servira de base à notre recherche, à notre dessin de caractère, de mise en page...\blank
\externalfigure [http://ospublish.constantvzw.org/images/var/resizes/Up-Pen-Down-December/P1130050.JPG] [width=10cm] \externalfigure [http://ospublish.constantvzw.org/images/var/resizes/Up-Pen-Down-December/conv_IMG_0949.jpg][width=10cm] \blank
\externalfigure[https://raw.github.com/no-feature/GRoland329/master/01.jpg][width=10cm] \externalfigure[https://raw.github.com/no-feature/GRoland329/master/02.jpg][width=10cm] \blank
\externalfigure[https://raw.github.com/no-feature/GRoland329/master/03.jpg][width=10cm]\blank
.......Bon mais tout ça c'était avant ! 
\page
{\tfc \ss GRoland 329}\blank
{\ss CONSTATS ET DÉSIRS}\blank
\setupcolumns[n=2, ntop=4]
     \startcolumns
                                    \startitemize 
                                         \item {Nos petits robots qui dessinent ont produit des sortes de cartographies}
                                         \item {Transition d'échelles}
                                         \item {Observation d'irrégularités et de formes intéressantes dans le trait}
                                         \item {Reproduction}
                                         \item {Agrandissement}
                                         \item {Déformation}
                                         \item {Notion temps dans la forme du trait (pointillés, épaisseur...)}
                                         \item {Fabrication d'un pantographe avec un moteur (on a remarqué que le moteur permettait un tracé plus fluide et qui ressemblait au 
                                                    plotter notamment dans l'observation des points des lignes)}
                                         \item {Pantographe : prolongement du geste, déformation du geste par les frottements dans les articulations du Groland, transition 
                                                   d'échelle, grossissement du dessin original.}
                                         \item {Utilisation du pantographe comme un outils de reproduction, et peut être envisager en outils de double productions (polygraphe).} 
                                         \item {une écriture en deux temps (polygraphe) créer un caractère d'un côté et voir ses défauts}         
                                   \stopitemize
          \stopcolumns  
          \page
          \externalfigure [../IMAGES/GR/Capture ecran IMG_1874/03.png]
          \externalfigure [../IMAGES/GR/Capture ecran IMG_1874/11.png]
          \externalfigure [../IMAGES/GR/Capture ecran IMG_1874/16.png]

\page
          {\tfc \ss Pantographe}\blank
          Le premier pantographe a été construit en 1603 par Christoph Scheiner1, un astronome allemand, qui utilisa l'instrument pour recréer des diagrammes. Un premier bras est fixe par rapport au support, le bras central est prolongé par un petit pointeur, et le dernier est muni d'un crayon. En déplaçant le pointeur sur le diagramme, une copie du diagramme est réalisée par le crayon sur une autre feuille de papier. La dimension de l'image produite peut être changée en modifiant la dimension du parallélogramme.
Une description (texte et dessin) du pantographe apparaît dans L'Encyclopédie de Diderot et D'Alembert, à la moitié du XVIIIe siècle2 :
« (Art du Dessein) le pantographe ou singe, est un instrument qui sert à copier le trait de toutes sortes de desseins et de tableaux,  à les réduire, si l'on veut, en grand ou en petit ; il est composé de quatre regles mobiles ajustées ensemble sur quatre pivots, et qui forment entr'elles un parallélogramme. A l'extrémité d'une de ces regles prolongées est une pointe qui parcourt tous les traits du tableau, tandis qu'un crayon fixé à l'extrémité d'une autre branche semblable, trace légèrement ces traits de même grandeur, en petit ou en grand, sur le papier ou plan quelconque, sur lequel on veut les rapporter… »
— Encyclopédie ou Dictionnaire raisonné des sciences, des arts et des métiers.
Un autre instrument ayant également pour objet de reproduire des figures avec changement d'échelle, l'eidographe, a été conçu par William Wallace en 1821.
http://fr.wikipedia.org/wiki/Pantographe(dessin)\blank
\setupcolumns[n=2]
       \startcolumns
       \externalfigure [../IMAGES/GR/From Book Pantographice seu ars delineandi, Page 29-Pantograph_by_Christoph_Scheiner.jpg]  "Pantograph"  de Christoph Scheiner, Page 29\blank
\externalfigure [../IMAGES/GR/image037.jpg] \blank
\externalfigure [../IMAGES/GR/image045.jpg]Schéma du pantographe classique de Christoph Scheiner\blank
\externalfigure [../IMAGES/GR/Pantographe-report-of-superintendent-1867.jpg]
 
 \stopcolumns
 \externalfigure [(http-_www.denshaotaku365.com_albums_choshi_dentetsu_________photos_4602015-img_0096.html).JPG]
          {\bf \tfc Ou sinon un pantographe c'est aussi çà !}
          \page         
\blank 
{\tfc \ss GRoland 329 •}\blank dans un premier temps à travers la reproduction d'un alphabet met en avant la dégradation, modification, transformation du trait, jusqu'à mettre en exergue les variations de formes dû aux changements d'échelles.
Face à l'utilisation quotidienne d'outils de reproduction de plus en plus performant, nous souhaitions expérimenter l'outil du pantographe où l'accident survient facilement grâce aux poids et aux frottements des matériaux utilisés (bois, feuille, moteur, roulettes, écrous, vis). Provocant ainsi une expérience sensible sur l'outils et son emploi. Le rapport au temps et à l'espace est finalement très vite perturber. Dans un temps de pratique où normalement le corps reste très distant de la production final, ici il joue un rôle majeur.Nous avons d'abord proposé un outil de reproduction et d'agrandissement. 
\blank {\tfa"Avec la gravure sur bois, on réussit pour la première fois à reproduire le dessin, bien longtemps avant que l'imprimerie permit la reproduction de l'écriture. On connaît les immenses transformations introduites dans la littérature par l'imprimerie, c'est-à dire par la reproduction technique de l'écriture"(1)}\blank
Ainsi la machine, par de multiple accidents, produit des copies déformant le dessin original.
En s'appuyant sur l'expérience du pantographe, nous cherchons à questionner la transition d'échelle et la notion de reproductibilité. A l'heure d'une reproductibilité de masse accrue, qu'en est-il de la présence du geste dans ces transitions d'échelle ?Comment le pantographe réintroduit-il l'accident à la manière du plotter ? Et comment l'accident laisse place à une trace toujours plus riche et exploitable ? 
La fractalité de la trace, à travers cette démarche, questionne la notion d'échelle (infiniment grand et infiniment petit). Comment la ligne d'une lettre peut-elle par un agrandissement devenir une possible ligne de cartographie ?
Outre cette notion d'agrandissement de la ligne, la translation proportionnelle que l'on obtient grâce à notre pantographe, nous amène à transposer le dessin d'une main à l'échelle du corps. Le dessin prend des proportions qui lui donne un nouveau statut car elle "rentre dans l'espace". \blank (1) : Walter Benjamin "L'oeuvre d'art à l'époque de sa reproductibilité technique" \blank
\setupcolumns[n=2]
       \startcolumns
                  \externalfigure [http://ospublish.constantvzw.org/images/var/albums/Up-Pen-Down-December/IMG_1840.JPG][width=7cm] \blank
                   \externalfigure [http://ospublish.constantvzw.org/images/var/resizes/Up-Pen-Down-December/IMG_1842.JPG][width=7cm] \blank
                    \externalfigure [http://ospublish.constantvzw.org/images/var/resizes/Up-Pen-Down-December/IMG_1843.JPG][width=7cm] \blank
                    \externalfigure [http://ospublish.constantvzw.org/images/var/resizes/Up-Pen-Down-December/IMG_2063.JPG][width=7cm] \blank
                    \externalfigure [http://ospublish.constantvzw.org/images/var/resizes/Up-Pen-Down-December/IMG_2064.JPG][width=7cm] \blank
                    \page
                           \stopcolumns
                   \externalfigure [../IMAGES/GR/Capture ecran IMG_1882/00.png][width=1cm] 
                                      \externalfigure [../IMAGES/GR/Capture ecran IMG_1882/03.png][width=1cm]                
                                      \externalfigure [../IMAGES/GR/Capture ecran IMG_1882/06.png][width=1cm]                   
                                      \externalfigure [../IMAGES/GR/Capture ecran IMG_1882/16.png][width=1cm] 
                                                                         \externalfigure [../IMAGES/GR/Capture ecran IMG_1882/20.png][width=1cm] 

\stoptext
\stopcomponent